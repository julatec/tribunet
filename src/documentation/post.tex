\documentclass{article}
\usepackage[utf8x]{inputenc}
\usepackage[english]{babel}
\usepackage[hyphens]{url}
\usepackage{booktabs}
\usepackage{natbib}
\usepackage{hyperref}
\usepackage{listings}
\bibliographystyle{plain}

\newcommand*{\fullref}[1]{\hyperref[{#1}]{\autoref*{#1} \nameref*{#1}}}

\title{Too much of the same \\
{\small Annotation Processor and the Adapter Design Pattern}}
\author{J Ulate C}
\date{\today}

\begin{document}

\maketitle

\section{The journey begins}

The first part of a journey might be the plan and the goal, but real life many times lacks of the those gifts. So, I started with taking a look on the kind of data that we require to handle.
The the proper place is the official documentation \cite{MinisteriodeHacienda2020PortalEstructuras}:

\begin{enumerate}
\item Estructuras XML y Anexos Version 1.0
\item Estructuras XML y Anexos Version 2.0
\item Estructuras XML y Anexos Version 3.0
\item Estructuras XML y Anexos Version 3.1
\item Estructuras XML y Anexos Version 4.0
\item Estructuras XML y Anexos Version 4.1
\item Estructuras XML y Anexos \textbf{Version 4.2}
\item Estructuras XML y Anexos \textbf{Version 4.3}
\end{enumerate}

There is nice a detail here, the \texttt{Version 4.2} has two sub-versions based on the year. Each one of these items contains a list of files with \texttt{XSD} files. 

\begin{itemize}
\item \texttt{Mensaje Hacienda\_v4.2.xsd}
\item \texttt{FacturaElectronica\_V.4.2.xsd}
\item \texttt{Mensaje Receptor\_v4.2.xsd}
\item \texttt{Nota Credito Electronica\_v4.2.xsd}
\item \texttt{Nota Debito Electronica\_v4.2.xsd}
\item \texttt{Tiquete Electronico\_v4.2.xsd}
\end{itemize}

The \texttt{Version 4.3} includes new members to the party, apart for the newer versions of the files above:

\begin{itemize}
\item \texttt{FacturaElectronicaCompra\_V4.3.xsd}
\item \texttt{FacturaElectronicaExportacion\_V4.3.xsd}
\end{itemize}

The first challenge start just after downloading the files. Taking a look on them, we can count more than 600 different entities. Probably too much to code by hand.

\section{Generating the entities from the \texttt{XSD} files}

After some research, \texttt{JAXB} \cite{Apache2020ApacheCxf-xjc-plugin} seems to be an old, and still valid, approach for these kind of situations. The \texttt{XSD} where placed on the \href{https://github.com/julatec/tribunet/tree/master/src/main/resources/xsd/tribunet}{\texttt{src/main/resources/xsd/tribunet/}} folder. For further details take a look on the \href{https://github.com/julatec/tribunet/blob/master/pom.xml}{\texttt{pom.xml}} file. On \fullref{lst:xsdgenerationsample} we can take a look how the code generation was configured.

\lstinputlisting[
language=XML,
label=lst:xsdgenerationsample,
caption={Option for code generation from \texttt{XSD} file}
]{listings/xsdgenerationsample.xml}

We are not going to place all the configurations and files generated here since there 

\section{Working the generated classes}

Once the files were generated we can see that there are many of them that corresponds to the \textit{same} entity, at least the same concept implemented in a similar way through different versions. The details are in \fullref{tbl:entities-repeated}.

% Please add the following required packages to your document preamble:
% \usepackage{booktabs}
\begin{table}[ht]
\begin{tabular}{@{}rl@{}}
\toprule
\textbf{Number of Entities} & \multicolumn{1}{c}{\textbf{Concept}} \\ \midrule
14                          & UbicacionType.java                   \\
14                          & TelefonoType.java                    \\
14                          & ReceptorType.java                    \\
14                          & ImpuestoType.java                    \\
14                          & IdentificacionType.java              \\
14                          & ExoneracionType.java                 \\
14                          & EmisorType.java                      \\
14                          & CodigoType.java                      \\
6                           & OtrosCargosType.java                 \\
6                           & DescuentoType.java                   \\
6                           & CodigoMonedaType.java                \\
4                           & ImpuestoResumenType.java             \\
3                           & TiqueteElectronico.java              \\
3                           & NotaDebitoElectronica.java           \\
3                           & NotaCreditoElectronica.java          \\
3                           & MensajeReceptor.java                 \\
3                           & MensajeHacienda.java                 \\
3                           & FacturaElectronica.java              \\
1                           & FacturaElectronicaExportacion.java   \\
1                           & FacturaElectronicaCompra.java        \\ \bottomrule
\end{tabular}
\caption{Number of repeated entities among different versions and documents}
\label{tbl:entities-repeated}
\end{table}

\medskip
\bibliography{references}
\end{document}
